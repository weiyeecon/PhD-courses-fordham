%EX TS-program = pdflatex
% !TEX encoding = UTF-8 Unicode

% This is a simple template for a LaTeX document using the "article" class.
% See "book", "report", "letter" for other types of document.

\documentclass[11pt]{article} % use larger type; default would be 10pt
\usepackage[utf8]{inputenc} % set input encoding (not needed with XeLaTeX)

%%% Examples of Article customizations
% These packages are optional, depending whether you want the features they provide.
% See the LaTeX Companion or other references for full information.
\usepackage{amsmath}
\makeatletter
\renewcommand*\env@matrix[1][*\c@MaxMatrixCols c]{%
	\hskip -\arraycolsep
	\let\@ifnextchar\new@ifnextchar
	\array{#1}}
\makeatother

\newcommand{\norm}[1]{\left\lVert#1\right\rVert}
%%% PAGE DIMENSIONS
\usepackage{geometry} % to change the page dimensions
\usepackage{listings}
\usepackage[dvipsnames]{xcolor}
\usepackage{marvosym}
\geometry{a4paper} % or letterpaper (US) or a5paper or....
% \geometry{margin=2in} % for example, change the margins to 2 inches all round
% \geometry{landscape} % set up the page for landscape
%   read geometry.pdf for detailed page layout information

\usepackage{graphicx} % support the \includegraphics command and options
% \usepackage[parfill]{parskip} % Activate to begin paragraphs with an empty line rather than an indent
\usepackage{amssymb}
\usepackage{mathrsfs}
%%% PACKAGES
\usepackage{booktabs} % for much better looking tables
\usepackage{array} % for better arrays (eg matrices) in maths
\usepackage{paralist} % very flexible & customisable lists (eg. enumerate/itemize, etc.)
\usepackage{verbatim} % adds environment for commenting out blocks of text & for better verbatim
\usepackage{subfig} % make it possible to include more than one captioned figure/table in a single float
% These packages are all incorporated in the memoir class to one degree or another...
\usepackage{pgfplots}
%%% HEADERS & FOOTERS
\usepackage{fancyhdr} % This should be set AFTER setting up the page geometry
\pagestyle{fancy} % options: empty , plain , fancy
\renewcommand{\headrulewidth}{0pt} % customise the layout...
\lhead{}\chead{}\rhead{}
\lfoot{}\cfoot{\thepage}\rfoot{}

%%% SECTION TITLE APPEARANCE
\usepackage{sectsty}
\allsectionsfont{\sffamily\mdseries\upshape} % (See the fntguide.pdf for font help)
% (This matches ConTeXt defaults)
\usepackage[thinc]{esdiff}
\usepackage{bbold}
\usepackage{MnSymbol,wasysym}
%%% ToC (table of contents) APPEARANCE
\usepackage[nottoc,notlof,notlot]{tocbibind} % Put the bibliography in the ToC
\usepackage[titles,subfigure]{tocloft} % Alter the style of the Table of Contents
\renewcommand{\cftsecfont}{\rmfamily\mdseries\upshape}
\renewcommand{\cftsecpagefont}{\rmfamily\mdseries\upshape} % No bold!

%%% END Article customizations

%%% The "real" document content comes below...

\title{Homework 1}
\author{Wei Ye\footnote{ 1st year PhD student in Economics Department at Fordham University. Email: wye22@fordham.edu}
    \\ ECON 7010- Microeconomics II}
\date{Due on Jan 26, 2022}
\begin{document}
\maketitle

\section{Question 1 -- 1.C.1}
\textbf{Solution:}

Since $(\mathscr{B},C(\cdot))$ satisfies WARP, and $C(\{x,y\})=\{x\}$, which means $x\succ y$. In the meanwhile,
$\{x,y\}\cap\{x,y,z\}=\{x,y\}$.
We discuss by four scenarios:
\begin{enumerate}[1)]
    \item If $C(\{x,y,z\})=\{z\}$, this means $z\succ x$ and $z\succ y$ It doesn't violate WARP of $(\mathscr{B},C(\cdot))$.
    \item If $C(\{x,y,z\})=\{x\}$, this means $x\succ y$ and $x\succ z$. By WARP, $\{x\}$ should be chosen in $C(\{x,y\})$,it is! 
    \item If $C(\{x,y,z\})=\{y\}$, this means $y\succ x$ and $y\succ z$, by WARP, $\{y\}$ should be chosen in $C(\{x,y\})$, but it's not. So it's a contradiction.
    \item If $C(\{x,y,z\})=\{x,z\}$, this means $\{x,z\}\succ y$, by WARP, x should be in $C(\{x,y\})$, it is!
\end{enumerate} 
For other two scenarios, they are related to y, but y should not be chosen as it contradicts to the statement that our choice structure is WARP.So we eliminate other  two scenarios.
Therefore, the only possible situations are $C(\{x,y,z\})=\{x\},=\{z\}, or =\{x,z\}$ when the choice structure satisfies WARP.\smiley


\section{Question 2-- 1.C.2}
\textbf{Proof:}
\begin{enumerate}[1):]
    \item Prove $\Rightarrow$:
    
        Since WARP is satisfied, and from the condition, $B\cap B'=\{x,y\}$, $x\in C(B)$, by WARP, $x\in C(B')$. As $y\in C(B')$ in condition, by WARP, $y\in C(B)$. Thus, we can make a conclusion $\{x,y\}\subset C(B)$ and $\{x,y\}\subset C(B')$.
    \item Prove $\Leftarrow$:

    When given listed conditions, as $\{x,y\}\cap\{x,y,z\}=\{x,y\}$, and $\{x,y\}\subset C(B)$ and $\{x,y\}\subset C(B')$, we can go back and get when $x\in C(B)$ and  $x\in C(B')$ as we defined in Definition 1.C.1 in textbook. Thus, WARP is statisfied for the choice structure.
\end{enumerate}
Combining there two direction proof, we can conclude WARP is equivalent to the property listed in the quesiton.\smiley


\section{Question 3-1.C.3}
\textbf{Solution:}

\begin{enumerate}[a).]
    \item The answer is yes. If there are less than two elements in the choice sets, like the example in the question, it has transitive property. Because $x\in C(B)$ and $y\in C(B)$, it means $x\succ^*y$.
    \item Proof:

            Suppose $\{x,y,z\} \subset  C(B)$, then $C(B)\neq \emptyset$. If $x\in C(\{x,z\})$, then $x\succ^*z$, if  $y\in C(\{z,y\})$, then $y\succ^*z$,  by the rationality, only x is chosen when the choice set is $\{x,y,z\}$, thus, $x\succ^*y\succ^*z$, and transitivity $x\succ^* y$  holds  in this case.\smiley
\end{enumerate}

\section{Question 4-- 1.D.3}
\textbf{Proof:}

As in the question, $X=\{x,y,z\}\subset (\mathscr{B},C(\cdot))$. We make our proof by contradiction, if $x\in C(X)$, since $\{x,z\}\subset X$, $x\in C(\{x,z\})$, but from the information we get,
$C(\{x,z\})=\{z\}\neq\{x\}$, it's a contradiction. Thus, the choice structure $(\mathscr{B},C(\cdot))$ violates  weak  axiom. \smiley
\end{document}






























































