%EX TS-program = pdflatex
% !TEX encoding = UTF-8 Unicode

% This is a simple template for a LaTeX document using the "article" class.
% See "book", "report", "letter" for other types of document.

\documentclass[11pt]{article} % use larger type; default would be 10pt

\usepackage[utf8]{inputenc} % set input encoding (not needed with XeLaTeX)

%%% Examples of Article customizations
% These packages are optional, depending whether you want the features they provide.
% See the LaTeX Companion or other references for full information.
\usepackage{amsmath}
\makeatletter
\renewcommand*\env@matrix[1][*\c@MaxMatrixCols c]{%
	\hskip -\arraycolsep
	\let\@ifnextchar\new@ifnextchar
	\array{#1}}
\makeatother

\newcommand{\norm}[1]{\left\lVert#1\right\rVert}
%%% PAGE DIMENSIONS
\usepackage{geometry} % to change the page dimensions
\usepackage{marvosym}
\geometry{a4paper} % or letterpaper (US) or a5paper or....
% \geometry{margin=2in} % for example, change the margins to 2 inches all round
% \geometry{landscape} % set up the page for landscape
%   read geometry.pdf for detailed page layout information

\usepackage{graphicx} % support the \includegraphics command and options
% \usepackage[parfill]{parskip} % Activate to begin paragraphs with an empty line rather than an indent
\usepackage{amssymb}
%%% PACKAGES
\usepackage{booktabs} % for much better looking tables
\usepackage{array} % for better arrays (eg matrices) in maths
\usepackage{paralist} % very flexible & customisable lists (eg. enumerate/itemize, etc.)
\usepackage{verbatim} % adds environment for commenting out blocks of text & for better verbatim
\usepackage{subfig} % make it possible to include more than one captioned figure/table in a single float
% These packages are all incorporated in the memoir class to one degree or another...
\usepackage{pgfplots}
%%% HEADERS & FOOTERS
\usepackage{fancyhdr} % This should be set AFTER setting up the page geometry
\pagestyle{fancy} % options: empty , plain , fancy
\renewcommand{\headrulewidth}{0pt} % customise the layout...
\lhead{}\chead{}\rhead{}
\lfoot{}\cfoot{\thepage}\rfoot{}

%%% SECTION TITLE APPEARANCE
\usepackage{sectsty}
\allsectionsfont{\sffamily\mdseries\upshape} % (See the fntguide.pdf for font help)
% (This matches ConTeXt defaults)
\usepackage[thinc]{esdiff}
\usepackage{bbold}
\usepackage{MnSymbol,wasysym}
%%% ToC (table of contents) APPEARANCE
\usepackage[nottoc,notlof,notlot]{tocbibind} % Put the bibliography in the ToC
\usepackage[titles,subfigure]{tocloft} % Alter the style of the Table of Contents
\renewcommand{\cftsecfont}{\rmfamily\mdseries\upshape}
\renewcommand{\cftsecpagefont}{\rmfamily\mdseries\upshape} % No bold!

%%% END Article customizations

%%% The "real" document content comes below...

\title{HW1}
\author{Wei Ye\footnote{ 1st year PhD student in Economics Department at Fordham University. Email: wye22@fordham.edu}
    \\ ECON 7910 Econometrics}
\date{Due on Sep 23, 2021}

\begin{document}
	\maketitle

\section{Question 1-- 2.1}
\textbf{Solution:}

\begin{enumerate}
    \item    \begin{itemize}
        \item \begin{equation*}
            \frac{\partial E(y|x_1,x_2)}{\partial x_1}=\beta_1+\beta_4 x_2
        \end{equation*}
        \item \begin{equation*}
            \frac{\partial E(y|x_1,x_2)}{\partial x_2}=\beta_2+2\beta_3x_2+\beta_4x_1
        \end{equation*}
    \end{itemize}
    \item $E(u|x_1,x_2)$ means error term u and corvariates are independ,i.e., $E(u|x_1,x_2)=0$. For $E(u|x_1,x_2,x_2^2,x_1x_2)$, if 
            we are give previous CE, then in this CE, $x_2^2$ and $x_1x_2$ are both redundent, because they can be expressed by $x_1$ and $x_2$. 
    \item Given part(b), $E(u|x_1,x_2)$ doesn't give any additional information about $Var(u|x_1,x_2)$. The only thing we can know for sure is this item is nonnegative. But for whetere it's constant or relies on other variables, aka, heterogeneity or homogenity, we have no information.
\end{enumerate}

\section{Question 2 -- 2.2 }
\textbf{Solution:}

\begin{enumerate}
    \item Since in this question, $\mu = E(x)$, thus:
    \begin{equation*}
        \frac{\partial E(y|x)}{\partial x}=\delta_1+2\delta_2x-2\mu=\delta_1+2\delta_2(x-\mu)
    \end{equation*}
    \item Take expectation on both sides:
    \begin{equation*}
        E(\frac{\partial E(y|x)}{\partial x})=\delta_1+2\delta_2E(x-E(x))=\delta_1+0=\delta_1
    \end{equation*}
    \item 
    \begin{align*}
        L(y|1,x)&=L(E(y)|1,x)\\
        &=L(\delta_0+\delta_1(x_1-\mu)+\delta_2(x_1-\mu)^2|x)\\
        &=\delta_0+\delta_1(x_1-\mu)+\delta_2L((x_1-\mu)^2|1,x)\\
        &= \alpha_0+\delta_1x_1
    \end{align*}
    For the first equation above, I borrow the result from the Property LP.5 of Appendix 2.A.3 directly. And the last equation, I intend to negelect
     some extra proof, because this equation is based on what I guess. The last $L(.)$ would be a constant, 
     so $\alpha_0= \delta_0-\delta_1 \mu + L(.)$, and it's still a constant number, i.e., it doesn't rely on x. 
\end{enumerate}

\section{Question 3 -- 2.4}
\textbf{Solution:}

Since the expectation of u given x v is the function of x and v, and u, v is uncorrelated with x, then, it's conveniently to kill x term in this conditional expectation.
So the CE becomes: $E(u|x,v)=E(u|v)$.
We assume that $E(u|v)=\rho_0+\rho_1v$, By LIE, $E(E(u|v))=0=\rho_0+\rho_1E(v)$. From the question, $E(v)=0$, then we can deduce that $\rho_0=0$. Thus, 
back to our previous equation, $E(u|v)=\rho_1v$. $\smiley$.

\section{Question 4 -- 2.7}
\textbf{Solution:}\footnote{I partly refer to the online resource about the beginning of this solution, because
I forgot to set up a more flexible function of y, aka, euqation (\ref{y equation}).}

Since $E(y|x,z)=g(x)+z\beta$, we can generalize y with corvariates x and z:
\begin{equation}
    y=g(x)+z\beta+\epsilon \label{y equation}
\end{equation}
Take expectation on both sides of (\ref{y equation}) on the condition of x:
\begin{equation} \label{y:expected:equation}
    E(y|x)=g(x)+\beta E(z|x) 
\end{equation}
(\ref{y equation})-(\ref{y:expected:equation}): 
\begin{equation}\label{Rearrange:equation}
    y-E(y|x)=z\beta-\beta E(z|x)+\epsilon=\beta(z-E(z|x))+\epsilon
\end{equation}
Subsitute $\tilde{y}$ and $\tilde{z}$ into equation (\ref{Rearrange:equation}):
\begin{equation}\label{tilde:equation}
    \tilde{y}=\beta \tilde{z}+\epsilon
\end{equation}
Take expectation of equation of (\ref{tilde:equation}) on variable $\tilde{z}$:
\begin{equation}\label{final:equation}
    E(\tilde{y}|\tilde{z})=\beta \tilde{z}+0=\beta\tilde{z}
\end{equation}
The reason why we get 0 from equation (\ref{final:equation}) is that the error term $\epsilon$ is uncorrelated with 
$\tilde{u}$, thus, $E(\epsilon|\tilde{z})=0$. 







































































\end{document}