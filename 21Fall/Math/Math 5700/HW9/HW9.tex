%EX TS-program = pdflatex
% !TEX encoding = UTF-8 Unicode

% This is a simple template for a LaTeX document using the "article" class.
% See "book", "report", "letter" for other types of document.

\documentclass[11pt]{article} % use larger type; default would be 10pt

\usepackage[utf8]{inputenc} % set input encoding (not needed with XeLaTeX)

%%% Examples of Article customizations
% These packages are optional, depending whether you want the features they provide.
% See the LaTeX Companion or other references for full information.
\usepackage{amsmath}
\makeatletter
\renewcommand*\env@matrix[1][*\c@MaxMatrixCols c]{%
	\hskip -\arraycolsep
	\let\@ifnextchar\new@ifnextchar
	\array{#1}}
\makeatother

\newcommand{\norm}[1]{\left\lVert#1\right\rVert}
%%% PAGE DIMENSIONS
\usepackage{geometry} % to change the page dimensions
\usepackage{marvosym}
\geometry{a4paper} % or letterpaper (US) or a5paper or....
% \geometry{margin=2in} % for example, change the margins to 2 inches all round
% \geometry{landscape} % set up the page for landscape
%   read geometry.pdf for detailed page layout information

\usepackage{graphicx} % support the \includegraphics command and options
% \usepackage[parfill]{parskip} % Activate to begin paragraphs with an empty line rather than an indent
\usepackage{amssymb}
%%% PACKAGES
\usepackage{booktabs} % for much better looking tables
\usepackage{array} % for better arrays (eg matrices) in maths
\usepackage{paralist} % very flexible & customisable lists (eg. enumerate/itemize, etc.)
\usepackage{verbatim} % adds environment for commenting out blocks of text & for better verbatim
\usepackage{subfig} % make it possible to include more than one captioned figure/table in a single float
% These packages are all incorporated in the memoir class to one degree or another...
\usepackage{pgfplots}
%%% HEADERS & FOOTERS
\usepackage{fancyhdr} % This should be set AFTER setting up the page geometry
\pagestyle{fancy} % options: empty , plain , fancy
\renewcommand{\headrulewidth}{0pt} % customise the layout...
\lhead{}\chead{}\rhead{}
\lfoot{}\cfoot{\thepage}\rfoot{}

%%% SECTION TITLE APPEARANCE
\usepackage{sectsty}
\allsectionsfont{\sffamily\mdseries\upshape} % (See the fntguide.pdf for font help)
% (This matches ConTeXt defaults)
\usepackage[thinc]{esdiff}
\usepackage{bbold}
\usepackage{MnSymbol,wasysym}
%%% ToC (table of contents) APPEARANCE
\usepackage[nottoc,notlof,notlot]{tocbibind} % Put the bibliography in the ToC
\usepackage[titles,subfigure]{tocloft} % Alter the style of the Table of Contents
\renewcommand{\cftsecfont}{\rmfamily\mdseries\upshape}
\renewcommand{\cftsecpagefont}{\rmfamily\mdseries\upshape} % No bold!

%%% END Article customizations

%%% The "real" document content comes below...

\title{HW9}
\author{Wei Ye\footnote{I worked on my assignment sololy. Email: wye22@fordham.edu}  	\\
	ECON 5700}
\date{Due on August 25, 2020.}


\begin{document}
	\maketitle
	\section{Question 1}
	\textbf{Solution:}
	
	In this question, it only asks us to prove whether it's orthogonal, not mentioning anything about basis, aks, linearly indepent. Life would be much easier.
	
	\begin{enumerate}
		\item Let $q_1=\begin{bmatrix}
			\frac{1}{\sqrt{2}}\\
			-\frac{1}{\sqrt{2}}
		\end{bmatrix}$, $q_2=\begin{bmatrix}
		\frac{1}{\sqrt{2}}\\
		\frac{1}{\sqrt{2}}
	\end{bmatrix}$. 
$q_1 \cdot q_2=\frac{1}{2}-\frac{1}{2}=0$. Interesting thing is that we use dot product to justify the orthogonality.

\item let $q_1=\begin{bmatrix}
	\frac{1}{3}\\
	\frac{1}{3}\\
	-\frac{1}{3}
\end{bmatrix}$, $q_2=\begin{bmatrix}
\frac{1}{2}\\-\frac{1}{2}\\0
\end{bmatrix}$, and $q_3=\begin{bmatrix}
\frac{1}{5}\\\frac{1}{5}\\\frac{2}{5}
\end{bmatrix}$.

Use dot product between $q_1, q_2$ and $q_3$: 
$$q_1\cdot q_2=\frac{1}{6}-\frac{1}{6}=0$$
$$q_1\cdot q_3=\frac{2}{15}-\frac{2}{15}=0$$
$$q_2\cdot q_3=\frac{1}{10}-\frac{1}{10}=0$$
Thus., it's orthogonal.
	\end{enumerate}
	
\section{Question 2}
\textbf{Solution:}

Let $x_1=\begin{bmatrix}
	0\\1\\1
\end{bmatrix}$, $x_2=\begin{bmatrix}
1\\0\\1
\end{bmatrix}$, and $x_3=\begin{bmatrix}
1\\1\\0
\end{bmatrix}$.
	
	
\textbf{Step1:} By the G-S process:

$$v_1=x_1=\begin{bmatrix}
	0\\1\\1
\end{bmatrix}$$
$$\vec{v_2}=\vec{x_2}-(\frac{\vec{v_1}\cdot \vec{x_2}}{\vec{v_1}\cdot \vec{v_1}})\vec{v_1}=\begin{bmatrix}
	1\\ -\frac{1}{2}\\\frac{1}{2}
\end{bmatrix}$$
	
$$\vec{v_3}=\vec{x_3}-(\frac{\vec{v_1}\cdot \vec{x_3}}{\vec{v_1}\cdot\vec{v_1}})\vec{v_1}-(\frac{\vec{v_2}\cdot \vec{x_3}}{\vec{v_2}\cdot \vec{v_2}})\vec{v_2}=\begin{bmatrix}
	\frac{2}{3}\\\frac{2}{3}\\-\frac{2}{3}
\end{bmatrix}$$

\textbf{Step2:} Normalize the vectors:

$$q_1=\frac{\vec{v_1}}{\norm{\vec{v_1}}}=\begin{bmatrix}
	0\\\frac{1}{\sqrt{2}}\\\frac{1}{\sqrt{2}}
\end{bmatrix}$$
	$$q_2=\frac{\vec{v_2}}{\norm{\vec{v_2}}}=\begin{bmatrix}
		\frac{\sqrt{6}}{2}\\-\frac{\sqrt{6}}{4}\\\frac{\sqrt{6}}{4}
	\end{bmatrix}$$
$$q_3=\frac{\vec{v_3}}{\norm{\vec{v_3}}}=\begin{bmatrix}
	\frac{4\sqrt{3}}{9}\\-\frac{4\sqrt{3}}{9}\\\frac{4\sqrt{3}}{9}
\end{bmatrix}$$

\section{Question 3}\label{Ques}
\textbf{Solution:}

Since $A=QR \longrightarrow Q^{T}A=Q^{T}QR=R$.

$$Q^{T}=\begin{bmatrix}
	\frac{2}{3}&\frac{1}{3}&-\frac{2}{3}\\
	\frac{1}{3}&\frac{2}{3}&\frac{2}{3}\\
	\frac{2}{3}&-\frac{2}{3}&\frac{1}{3}
\end{bmatrix}$$
$$R=Q^{T}A=\begin{bmatrix}
	\frac{2}{3}&\frac{1}{3}&-\frac{2}{3}\\
	\frac{1}{3}&\frac{2}{3}&\frac{2}{3}\\
	\frac{2}{3}&-\frac{2}{3}&\frac{1}{3}
\end{bmatrix} 
\begin{bmatrix}
	2&8&2\\
	1&7&-1\\
	-2&-2&1
\end{bmatrix}=\begin{bmatrix}
1&9&0\\
0&6&\frac{2}{3}\\
0&0&\frac{7}{3}
\end{bmatrix}$$

\section{Question 4}
\textbf{Solution:}
The same with \ref{Ques}:

$$R=Q^{T}A=\begin{bmatrix}
	\sqrt{6}&2\sqrt{6}\\
	0&\sqrt{3}
\end{bmatrix}$$

\section{Question 5}
\textbf{Solution:}

$$\det(A-\lambda I)=(1+\lambda)^2-9=0$$
$\lambda_1=2, \lambda_2=-4$.
\begin{itemize}
	\item When $\lambda=2:$
	$$|A-\lambda I|=\begin{bmatrix}[cc|c]
		1&-1&0\\
		0&0&0
	\end{bmatrix}$$
Thus, $x_1=x_2$, and the eigenvector in this case is $\begin{bmatrix}
	1\\
	1
\end{bmatrix}$.

\item When $\lambda=-4:$
$$|A-\lambda I|=\begin{bmatrix}[cc|c]
	1&1&0\\
	0&0&0
\end{bmatrix}$$
In this case the eigenvector is $\begin{bmatrix}
	1\\-1
\end{bmatrix}$.

Normalize the eigenvectors to obtain P:

$$P=\begin{bmatrix}
	\frac{1}{\sqrt{2}}&\frac{1}{\sqrt{2}}\\
	\frac{1}{\sqrt{2}}&-\frac{1}{\sqrt{2}}
\end{bmatrix}$$

$$D=P^{T}AP=\begin{bmatrix}
	\frac{1}{\sqrt{2}}&\frac{1}{\sqrt{2}}\\
\frac{1}{\sqrt{2}}&-\frac{1}{\sqrt{2}}
\end{bmatrix}
	\begin{bmatrix}
		2&0\\
		0&-4
	\end{bmatrix}
\begin{bmatrix}
		\frac{1}{\sqrt{2}}&\frac{1}{\sqrt{2}}\\
	\frac{1}{\sqrt{2}}&-\frac{1}{\sqrt{2}}
\end{bmatrix}$$
	
Leave the final result intentionally. 
	
\end{itemize}

\section{Question 6}
\textbf{Solution:}

\begin{enumerate}
	\item For the first matrix:
	$$\begin{bmatrix}
		1&2&3&-1\\
		2&6&3&0\\
		0&6&-6&7\\
		-1&-2&-9&0
	\end{bmatrix}\xrightarrow[R_2-2R_1]{R_4-(-1)R_1}
\begin{bmatrix}
	1&2&3&-1\\
	0&2&-3&2\\
	0&6&-6&7\\
	0&0&-6&-1
\end{bmatrix}\xrightarrow{R_3-3R_2}\begin{bmatrix}
1&2&3&-1\\
0&2&-2&2\\
0&0&3&1\\
0&0&-6&-1
\end{bmatrix}\xrightarrow{R_4-(-2)R_3}\begin{bmatrix}
1&2&3&-1\\
0&2&-3&2\\
0&0&3&1\\
0&0&0&1
\end{bmatrix}=U$$

$$L=\begin{bmatrix}
	1&0&0&0\\
	2&1&0&0\\
	0&3&1&0\\
	-1&0&-2&1
\end{bmatrix}$$

\item For the second matrix (example):
$$\begin{bmatrix}
	2&2&2&1\\
	-2&4&-1&2\\
	4&4&7&3\\
	6&9&5&8
\end{bmatrix}\xrightarrow[R_2-(-1)R_1\\
R_3-2R_1]{R_4-3R_1}\begin{bmatrix}
	2&2&2&1\\
	0&6&1&3\\
	0&0&3&1\\
	0&3&-1&5
\end{bmatrix}\xrightarrow{R_4-\frac{1}{2}R_2}\begin{bmatrix}
2&2&2&1\\
0&6&1&3\\
0&0&3&1\\
0&0&\frac{2}{3}&\frac{2}{7}
\end{bmatrix}\xrightarrow{R_4-(-\frac{1}{2})R_3}\begin{bmatrix}
	2&2&2&1\\
	0&6&1&3\\
	0&0&3&1\\
	0&0&0&4
\end{bmatrix}=U$$
$$L=\begin{bmatrix}
	1&0&0&0\\
	-1&1&0&0\\
	2&0&1&0\\
	3&\frac{1}{2}&-\frac{1}{2}&1
\end{bmatrix}$$


\section{Question 7}
\textbf{Solution:}

Since the least square method is not required, so I use conventional way to solve this problem. It's not about the approximation, but accurate solution. Once I get the solution, I will check how to make approximation with OLS method. 

$$\begin{bmatrix}[ccc|c]
	1&1&-2&2\\
	0&-1&2&6\\
	3&2&-1&11
\end{bmatrix}=\begin{bmatrix}[ccc|c]
1&1&-1&2\\
0&1&2&-6\\
0&0&0&-1
\end{bmatrix}$$
It's actually not solvable, so we have to use approximation (I KNOW IT, THIS IS THE REASON.).

\section{Question 8}
\textbf{Solution:}

\begin{align*}
	A^+&=(A^{T}A)^{-1}A^{T}\\
	&=(\begin{bmatrix}
		1&1&0&1\\
		0&0&1&1\\
		0&1&1&1
	\end{bmatrix}\begin{bmatrix}
	1&0&0\\
	1&0&1\\
	0&1&1\\
	1&1&1
\end{bmatrix})^{-1}\begin{bmatrix}
1&1&0&0\\
0&0&1&1\\
0&1&1&1
\end{bmatrix}\\
&=\begin{bmatrix}
	\frac{1}{3}&-\frac{1}{3}&-\frac{2}{3}\\
	\frac{1}{3}&\frac{5}{3}&-\frac{4}{3}\\
	-\frac{2}{3}&-\frac{4}{3}&\frac{5}{3}
\end{bmatrix}
\end{align*}


\section{Question 9}
\textbf{Solution:}

\begin{itemize}
	\item  Step 1:
	\begin{align*}
		A^{T}A&=\begin{bmatrix}
			1&1&0\\1&0&1
		\end{bmatrix}\begin{bmatrix}
		1&1\\
		1&0\\
		0&1
	\end{bmatrix}\\
&=\begin{bmatrix}
	2&1\\
	1&2
\end{bmatrix}
	\end{align*}

\item Step 2- Calculate the eigenvalues:
\begin{align*}
	|A^{T}A-\lambda I|&=\begin{bmatrix}
		2-\lambda&1\\
		1&2-\lambda
	\end{bmatrix}\\
&=\lambda^2-4\lambda+3\\
&=0
\end{align*}
Thus, $lambda_1=1, \lambda_2=3$.

\begin{itemize}
	\item If $\lambda=1$:
	$$\begin{bmatrix}
		1&1\\
		1&1
	\end{bmatrix} \longrightarrow 
\begin{bmatrix}
	1&1\\
	0&0
\end{bmatrix}$$
Thus, the eigenvector in this case is: $\begin{bmatrix}
	1\\-1
\end{bmatrix}$.
\item If  $\lambda=3$

$$\begin{bmatrix}
	-1&1\\
	1&-1
\end{bmatrix}\longrightarrow \begin{bmatrix}
-1&1\\
0&0
\end{bmatrix}$$
Thus, the eigenvector is $\begin{bmatrix}
	1\\1
\end{bmatrix}$
\end{itemize}
Thus 
$$\vec{v}=\begin{bmatrix}
	\frac{1}{\sqrt{2}}&\frac{1}{\sqrt{2}}\\
	-\frac{1}{\sqrt{2}}&\frac{1}{\sqrt{2}}
\end{bmatrix}$$

$$\Sigma =\begin{bmatrix}
	1&0&0\\
	0&3&0\\
	0&0&0
\end{bmatrix}$$
\item Step 3- Calculate u:
\begin{align*}
u_1&=1\cdot \begin{bmatrix}
	1&1\\1&0\\0&1
\end{bmatrix}\begin{bmatrix}
\frac{1}{\sqrt{2}}\\
-\frac{1}{\sqrt{2}}
\end{bmatrix}\\
&=\begin{bmatrix}
	0\\\frac{1}{\sqrt{2}}\\-\frac{1}{\sqrt{2}}
\end{bmatrix}
\end{align*}
\begin{align*}
	u_2&=\frac{1}{\sqrt{3}}\begin{bmatrix}
		1&1\\1&0\\0&1
	\end{bmatrix}\begin{bmatrix}
	\frac{1}{\sqrt{2}}\\\frac{1}{\sqrt{2}}
\end{bmatrix}\\
	&= \frac{1}{\sqrt{3}}\begin{bmatrix}
		\sqrt{2}\\\frac{1}{\sqrt{2}}\\\frac{1}{\sqrt{2}}
\end{bmatrix}\\
&=\begin{bmatrix}
	\frac{\sqrt{6}}{3}\\
	\frac{\sqrt{6}}{6}\\
	\frac{\sqrt{6}}{6}
\end{bmatrix}
\end{align*}


\item Step 4- A conclusion:

\begin{align*}
	A&=U\Sigma V^{T}\\
	&=\begin{bmatrix}
		0&\frac{\sqrt{6}}{3}\\
		\frac{1}{\sqrt{2}}& \frac{\sqrt{6}}{6}\\
		-\frac{1}{\sqrt{2}}&\frac{\sqrt{6}}{6}
	\end{bmatrix}\begin{bmatrix}
	1&0&0\\
	0&3&0\\
	0&0&0
\end{bmatrix}\begin{bmatrix}
\frac{1}{\sqrt{2}}&-\frac{1}{\sqrt{2}}\\
\frac{1}{\sqrt{2}}&\frac{1}{\sqrt{2}}
\end{bmatrix}
\end{align*}
\end{itemize}




\end{enumerate}

	
	
	
	
	
	
	
	
	
	
	
	
	
	
	
	
	
	
	
	
	
	
	
	
	
	
	
	
	
	
	
	
	
	
	
	
	
	
	
	
	
	
	
	
	
	
	
	
	
	
	
	
	
	
	
	
	
	
	
	
	
	
	
	
	
	
	
	
	
	
	
	
	
	
	
	
	
	
	
\end{document}