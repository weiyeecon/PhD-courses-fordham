%EX TS-program = pdflatex
% !TEX encoding = UTF-8 Unicode

% This is a simple template for a LaTeX document using the "article" class.
% See "book", "report", "letter" for other types of document.

\documentclass[11pt]{article} % use larger type; default would be 10pt

\usepackage[utf8]{inputenc} % set input encoding (not needed with XeLaTeX)

%%% Examples of Article customizations
% These packages are optional, depending whether you want the features they provide.
% See the LaTeX Companion or other references for full information.
\usepackage{amsmath}
\makeatletter
\renewcommand*\env@matrix[1][*\c@MaxMatrixCols c]{%
	\hskip -\arraycolsep
	\let\@ifnextchar\new@ifnextchar
	\array{#1}}
\makeatother
%%% PAGE DIMENSIONS
\usepackage{geometry} % to change the page dimensions
\usepackage{marvosym}
\geometry{a4paper} % or letterpaper (US) or a5paper or....
% \geometry{margin=2in} % for example, change the margins to 2 inches all round
% \geometry{landscape} % set up the page for landscape
%   read geometry.pdf for detailed page layout information

\usepackage{graphicx} % support the \includegraphics command and options
% \usepackage[parfill]{parskip} % Activate to begin paragraphs with an empty line rather than an indent
\usepackage{amssymb}
%%% PACKAGES
\usepackage{booktabs} % for much better looking tables
\usepackage{array} % for better arrays (eg matrices) in maths
\usepackage{paralist} % very flexible & customisable lists (eg. enumerate/itemize, etc.)
\usepackage{verbatim} % adds environment for commenting out blocks of text & for better verbatim
\usepackage{subfig} % make it possible to include more than one captioned figure/table in a single float
% These packages are all incorporated in the memoir class to one degree or another...
\usepackage{pgfplots}
%%% HEADERS & FOOTERS
\usepackage{fancyhdr} % This should be set AFTER setting up the page geometry
\pagestyle{fancy} % options: empty , plain , fancy
\renewcommand{\headrulewidth}{0pt} % customise the layout...
\lhead{}\chead{}\rhead{}
\lfoot{}\cfoot{\thepage}\rfoot{}

%%% SECTION TITLE APPEARANCE
\usepackage{sectsty}
\allsectionsfont{\sffamily\mdseries\upshape} % (See the fntguide.pdf for font help)
% (This matches ConTeXt defaults)
\usepackage[thinc]{esdiff}
\usepackage{bbold}
\usepackage{MnSymbol,wasysym}
%%% ToC (table of contents) APPEARANCE
\usepackage[nottoc,notlof,notlot]{tocbibind} % Put the bibliography in the ToC
\usepackage[titles,subfigure]{tocloft} % Alter the style of the Table of Contents
\renewcommand{\cftsecfont}{\rmfamily\mdseries\upshape}
\renewcommand{\cftsecpagefont}{\rmfamily\mdseries\upshape} % No bold!

%%% END Article customizations

%%% The "real" document content comes below...

\title{HW6}
\author{Wei Ye\footnote{I worked on my assignment sololy. Email: wye22@fordham.edu}  	\\
	ECON 5700}
\date{Due on August 20, 2020.}


\begin{document}
	\maketitle
	\section{Question 1}
	\textbf{Solution:}

Let $P=
\begin{pmatrix}
	0&-1\\
	1&0
\end{pmatrix}$. By deduction, we can get:

$$P^2=\begin{pmatrix}
	-1&0\\
	0&-1
\end{pmatrix}$$
$$P^3=\begin{pmatrix}
	0&1\\
	-1&0
\end{pmatrix}$$
$$P^4=\begin{pmatrix}
	1&0\\
	0&1
\end{pmatrix}$$
$$P^5=\begin{pmatrix}
	0&-1\\
	1&0
\end{pmatrix}$$
	Thus, if the remainder of  k divide 4 is 0, then the format of the matrix is like $P^4$, if the remainder is 1, the matrix is like $P^1$. If the remainder is 2, the format is $P^2$. And if the remainder is 3, like $P^3$.$\smiley$


\section{Question 2}
\textbf{Solution:}

Since $(A+B)^2=(A+B)(A+B)=A(A+B)+B(A+B)=A^2+AB+BA+B^2$, if and only if $AB+BA=2AB \longrightarrow  BA=AB$, then the equation holds. Otherwise, we can't say the equation holds.

\section{Question 3}
\textbf{Solution:}

It's dependent. 
let the coeffieient of the first matrix is a and the coefficient of the second matrix is b. We need to prove that:
$$a\begin{bmatrix}
	1&2\\
	4&3
\end{bmatrix}	+b\begin{bmatrix}
2&1\\
-1&0
\end{bmatrix}=\begin{bmatrix}
-1&-1\\
-1&-1
\end{bmatrix}$$
Solve this, we can get $a=-\frac{1}{3}\neq 0, b=-\frac{1}{3}\neq0$. Thus, these matrices are dependent. 

\section{Question 4}
\textbf{Solution:}

\begin{enumerate}
	\item 
	$$\begin{bmatrix}[ccc|ccc]
		2&3&0&1&0&0\\
		1&-2&-1&0&1&0\\
		2&0&-1&0&0&1
	\end{bmatrix}=
\begin{bmatrix}[ccc|ccc]
	1&0&0&2&3&-3\\
	0&1&0&-1&-2&2\\
	0&0&1&4&6&7
\end{bmatrix}$$
\item 
For this matrix the determinant is 0, thus, the inverse matrix doesn't exist. 

\end{enumerate}
	
\section{Question 5}
\textbf{Solution:}
\begin{enumerate}
	
\item 
$$ \frac{1}{8-7}
\begin{bmatrix}
	2&-7\\
	-1&4
\end{bmatrix}=\begin{bmatrix}
\frac{1}{4}&-\frac{7}{8}\\
-\frac{1}{8}&\frac{1}{2}
\end{bmatrix}$$
\item 
The inverse matrix is:

$$\begin{bmatrix}
	0&\frac{1}{2}\\
	-\frac{1}{2}&1
\end{bmatrix}$$

\item Since this matrix $3\cdot 4-4\cdot 6=0$, this matrix doesn't have inverse matrix. 

\item The inverse matrix is:
$$\begin{bmatrix}
	0 &-1\\
	1&0
\end{bmatrix}$$


\end{enumerate}

\section{Question 6}

\textbf{Solution:}

Af ter matrix elimination, the matrix A would be:
$$\begin{bmatrix}
	1&1&0&1\\
	0&1&-1&1\\
	0&0&0&1
\end{bmatrix}$$
Thus, the $row(A)=span\{[1 \ 1 \ 0 \ 1],[0 \ 1\  -1 \ 1],[0 \ 0 \ 0 \ 1] \}$,  
To find the col(a), we need to transpose A first and make elimination, get:

$$\begin{bmatrix}
	1&0&0\\
	0&1&1\\
	0&0&-2\\
	0&0&0
\end{bmatrix}$$
We get the $row(A^T)=span\{[1\ 0\ 0],[0\ 1\ 1],[0\ 0\ -2]\}$. Transpose this to get $col(A)=\{
[1\ 0\ 0]^T,[0\ 1\ 1]^T,[0\ 0\ -2]^T\}$

To get null(A), we need to begin with RE(A):
thus, $x_1=-x_2=-x_3, x_4=0$, let $x_3=x_2=t, x_1=-t$.
$$\vec{x}=\begin{bmatrix}
	-t\\
	t\\
	t
\end{bmatrix}=t\cdot \begin{bmatrix}
-1\\
1\\
1
\end{bmatrix}$$
Thus, the null(A)=$ t \cdot \begin{bmatrix}
	-1\\
	1\\
	1
\end{bmatrix}$, for $t\in \mathcal{R}$.

\section{Question 7}
\textbf{Solution:}

$$RE(A)=\begin{bmatrix}
	1&0&-1\\0&1&2
\end{bmatrix}$$
$row(A)=span\{[1\ 0\ -1],[0\ 1\ 2]\}$
Assume the coefficient of the first matrix is a, b for the second, c for w. Solving this euqation for the matrix independence, $a=b=0=c$. Thus, it can't be expressed. 

Follwing the method in Question 7, we can get the col(A)=$span\{[1\ 1]^T,[0 \ 1]^T\}$

$$3\cdot \begin{bmatrix}
	1\\
	1
\end{bmatrix}+(-1)\cdot \begin{bmatrix}
0\\1
\end{bmatrix}=\begin{bmatrix}
3\\2
\end{bmatrix}$$
Thus, b in col(A).


\end{document}