%EX TS-program = pdflatex
% !TEX encoding = UTF-8 Unicode

% This is a simple template for a LaTeX document using the "article" class.
% See "book", "report", "letter" for other types of document.

\documentclass[11pt]{article} % use larger type; default would be 10pt

\usepackage[utf8]{inputenc} % set input encoding (not needed with XeLaTeX)

%%% Examples of Article customizations
% These packages are optional, depending whether you want the features they provide.
% See the LaTeX Companion or other references for full information.
\usepackage{amsmath}
\makeatletter
\renewcommand*\env@matrix[1][*\c@MaxMatrixCols c]{%
	\hskip -\arraycolsep
	\let\@ifnextchar\new@ifnextchar
	\array{#1}}
\makeatother
%%% PAGE DIMENSIONS
\usepackage{geometry} % to change the page dimensions
\usepackage{marvosym}
\geometry{a4paper} % or letterpaper (US) or a5paper or....
% \geometry{margin=2in} % for example, change the margins to 2 inches all round
% \geometry{landscape} % set up the page for landscape
%   read geometry.pdf for detailed page layout information

\usepackage{graphicx} % support the \includegraphics command and options
% \usepackage[parfill]{parskip} % Activate to begin paragraphs with an empty line rather than an indent
\usepackage{amssymb}
%%% PACKAGES
\usepackage{booktabs} % for much better looking tables
\usepackage{array} % for better arrays (eg matrices) in maths
\usepackage{paralist} % very flexible & customisable lists (eg. enumerate/itemize, etc.)
\usepackage{verbatim} % adds environment for commenting out blocks of text & for better verbatim
\usepackage{subfig} % make it possible to include more than one captioned figure/table in a single float
% These packages are all incorporated in the memoir class to one degree or another...
\usepackage{pgfplots}
%%% HEADERS & FOOTERS
\usepackage{fancyhdr} % This should be set AFTER setting up the page geometry
\pagestyle{fancy} % options: empty , plain , fancy
\renewcommand{\headrulewidth}{0pt} % customise the layout...
\lhead{}\chead{}\rhead{}
\lfoot{}\cfoot{\thepage}\rfoot{}

%%% SECTION TITLE APPEARANCE
\usepackage{sectsty}
\allsectionsfont{\sffamily\mdseries\upshape} % (See the fntguide.pdf for font help)
% (This matches ConTeXt defaults)
\usepackage[thinc]{esdiff}
\usepackage{bbold}
\usepackage{MnSymbol,wasysym}
%%% ToC (table of contents) APPEARANCE
\usepackage[nottoc,notlof,notlot]{tocbibind} % Put the bibliography in the ToC
\usepackage[titles,subfigure]{tocloft} % Alter the style of the Table of Contents
\renewcommand{\cftsecfont}{\rmfamily\mdseries\upshape}
\renewcommand{\cftsecpagefont}{\rmfamily\mdseries\upshape} % No bold!

%%% END Article customizations

%%% The "real" document content comes below...

\title{HW5}
\author{Wei Ye\footnote{I worked on my assignment sololy. Email: wye22@fordham.edu}  	\\
	ECON 5700}
\date{Due on August 20, 2020.}


\begin{document}
	\maketitle
	\section{Question 1}
	\textbf{Solution:}
\begin{enumerate}
	\item $V=
	\begin{bmatrix}
		-1\\3
	\end{bmatrix}$, $u=
\begin{bmatrix}
	2\\1
\end{bmatrix}$. The projection of v onto u:

$$\begin{bmatrix}
	\frac{2}{5}\\
	\frac{1}{5}
\end{bmatrix}$$
\item $v=
\begin{bmatrix}
	1&2&3
\end{bmatrix}$, $u=e_3=
\begin{bmatrix}
	0&0&0\\
	0&0&0\\
	0&0&1
\end{bmatrix}$, the projection of v onto u:
$$\begin{bmatrix}
	0&0&0\\
	0&0&0\\
	0&0&3
\end{bmatrix}$$
\item $v=
\begin{bmatrix}
	1\\2\\3
\end{bmatrix}$, $u=
\begin{bmatrix}
	\frac{1}{2}&\frac{1}{2}&\frac{1}{\sqrt{2}}
\end{bmatrix}$. The projection of v onto u is:
$$\begin{bmatrix}
	\frac{3+3\sqrt{2}}{4}& \frac{3+3\sqrt{2}}{4} &\frac{3\sqrt{2}+6}{4}
\end{bmatrix}$$


\end{enumerate}




\section{Question 2}

\textbf{Solution:}


At the points $P=(-1,5,0)$ and $Q=(2,1,1)$, a direction vector through the two points are $v=<3,-4,1>$. Thus, the line of this vector is $r=r_0+t(r_1-r_0)$.

\begin{align*}
<x,y,z>&= \overrightarrow{OP}+t(\overrightarrow{OQ}-\overrightarrow{OP})\\
				&= (-1,5,0)+t(3,-4,1)
\end{align*}
This is one possible line. $\smiley$
	
\section{Question 3}

\textbf{Solution:}

The reason why I think w is given variable is that if it's not, then 4 unknown variables, but 3 equations, which make our equations unsolvable.

$$
\begin{pmatrix}[ccc|c]
	-1 &-1& 2&1-w\\
	-2 &-1 & 3&3-2w\\
	1&-1&0&-3+w
\end{pmatrix}
$$

$$
\begin{pmatrix}[ccc|c]
	1&-1&0&3+w\\
	0&-1&1&1\\
	0&0&0&-2
\end{pmatrix}
$$ 
This question is unsolvable. $\frownie{}$ 

\section{Question 4}
\textbf{Solution:}

$$\begin{pmatrix}[ccc|c]
	1&-1&2&3\\
	1&2&-1&-3\\
	0&2&-2&1
\end{pmatrix}
$$

$$\begin{pmatrix}[ccc|c]
	1&-1&3&3\\
	0&1&-1&-2\\
	0&0&0&-3
\end{pmatrix}
$$

Again, this system is inconsistent, thus unsolvable. $\frownie{}$

\section{Question 5}
\textbf{Solution:}

$$\begin{pmatrix}[ccc|c]
	1&1&-2&4\\
	1&3&-1&7\\
	2&1&-5&7
\end{pmatrix}
$$
After several elimination:

$$\begin{pmatrix}[ccc|c]
	1&0&0&0\\
	0&1&0&2\\
	0&0&1&-1
\end{pmatrix}
$$
Thus, $x=0, y=2, z=-1$.	 $\smiley{}$

\section{Question 6}
\textbf{Solution:}

$$\begin{pmatrix}
	1&-2&0&3&2\\
	0&5&1&-6&-2\\
	0&10&2&-12&-4\\
	0&-5&-1&6&2
\end{pmatrix}
$$
After sevral elimination process, the matrix is: 
$$\begin{pmatrix}
	1&-2&0&3&2\\
	0&5&1&-6&2\\
	0&0&0&0&0\\
	0&0&0&0&0
\end{pmatrix}$$
Therefore, the rank of this matrix is 2. $\smiley{}$

\section{Question 7}
\textbf{Solution:}

To justify the two matrices are equivalent or not, we need to make some eliminations on these two matrices, respectively. 
\begin{enumerate}
	\item This matrix would be: 
	$$\begin{pmatrix}
		1&0&4\\
		0&1&-3\\
		0&0&17
	\end{pmatrix}$$
	\item The second matrix would be:
	$$\begin{pmatrix}
		1&0&-1\\
		0&1&2\\
		0&0&1
	\end{pmatrix}$$
\end{enumerate}

\textbf{Remark:} If 17 in the first matrix can be extracted out, then after elimination, it becomes identity matrix. The second matrix can also be identity matrix without plugging anything out. But $17*\mathbb{1}?=\mathbb{1}$ ? \textcolor{blue}{Check!} \textcolor{red}{Yes, it is. The two matrix is equivalent, because, they can be transformed into identity matrix. }

\section{Question 8}
\textbf{Solution:}

$$\begin{pmatrix}[ccc|c]
	8&-18&1&35-3w\\
	2&-4&0&11-w\\
	3&-7&1&10-w
\end{pmatrix}$$

After elimination of this matrix:
$$\begin{pmatrix}[ccc|c]
	1&-2&0&\frac{2}{11}-\frac{w}{2}\\
	0&-1&-2&-\frac{13}{2}+\frac{w}{2}\\
	0&0&5&4+2w
\end{pmatrix}$$
	Thus, the equation would be: $x=\frac{153}{10}-\frac{31}{10}w$, $y=\frac{49}{10}-\frac{13}{10}w$, and $z=\frac{4}{5}+\frac{2}{5}w$.\textcolor{blue}{The result is weird, check later. }\textcolor{red}{I decide not to check again. Need to check with the solution sheet later.}
	
\section{Question 9}
\textbf{Solution:}

$u=
\begin{bmatrix}
	5\\
	1\\
	3
\end{bmatrix}$, $v=
\begin{bmatrix}
	3\\1\\-1
\end{bmatrix}$, $w=
\begin{bmatrix}
	7\\5\\8
\end{bmatrix}$, and $x=
\begin{bmatrix}
	x_1\\x_2\\x_3
\end{bmatrix}$. 

\begin{enumerate}
\item $$uv'=
\begin{bmatrix}
	15&5&-5\\
	3&1&-1\\
	9&3&-3
\end{bmatrix}$$
\item $$uw'=
\begin{bmatrix}
	35&25&40\\
	7&5&8\\
	21&15&24
\end{bmatrix}$$
\item $$xx'=
\begin{bmatrix}
	x_1^2&x_1x_2&x_1x_3\\
	x_2x_1&x_2^2&x_2x_3\\
	x_3x_1&x_3x_2&x_3^2
\end{bmatrix}$$
\item $$v'u=13$$
\item $$u'v=13$$
\item $$w'x=7x_1+5x_2+8x_3$$
\item $$u'u=35$$
\item $$x'x=x_1^2+x_2^2+x_3^2$$
\end{enumerate}

	
	
	
	
	
	
\end{document}