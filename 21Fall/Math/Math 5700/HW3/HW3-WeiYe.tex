%EX TS-program = pdflatex
% !TEX encoding = UTF-8 Unicode

% This is a simple template for a LaTeX document using the "article" class.
% See "book", "report", "letter" for other types of document.

\documentclass[11pt]{article} % use larger type; default would be 10pt

\usepackage[utf8]{inputenc} % set input encoding (not needed with XeLaTeX)

%%% Examples of Article customizations
% These packages are optional, depending whether you want the features they provide.
% See the LaTeX Companion or other references for full information.
\usepackage{amsmath}
%%% PAGE DIMENSIONS
\usepackage{geometry} % to change the page dimensions
\geometry{a4paper} % or letterpaper (US) or a5paper or....
% \geometry{margin=2in} % for example, change the margins to 2 inches all round
% \geometry{landscape} % set up the page for landscape
%   read geometry.pdf for detailed page layout information

\usepackage{graphicx} % support the \includegraphics command and options
% \usepackage[parfill]{parskip} % Activate to begin paragraphs with an empty line rather than an indent
\usepackage{amssymb}
%%% PACKAGES
\usepackage{booktabs} % for much better looking tables
\usepackage{array} % for better arrays (eg matrices) in maths
\usepackage{paralist} % very flexible & customisable lists (eg. enumerate/itemize, etc.)
\usepackage{verbatim} % adds environment for commenting out blocks of text & for better verbatim
\usepackage{subfig} % make it possible to include more than one captioned figure/table in a single float
% These packages are all incorporated in the memoir class to one degree or another...
\usepackage{pgfplots}
%%% HEADERS & FOOTERS
\usepackage{fancyhdr} % This should be set AFTER setting up the page geometry
\pagestyle{fancy} % options: empty , plain , fancy
\renewcommand{\headrulewidth}{0pt} % customise the layout...
\lhead{}\chead{}\rhead{}
\lfoot{}\cfoot{\thepage}\rfoot{}

%%% SECTION TITLE APPEARANCE
\usepackage{sectsty}
\allsectionsfont{\sffamily\mdseries\upshape} % (See the fntguide.pdf for font help)
% (This matches ConTeXt defaults)
\usepackage[thinc]{esdiff}
%%% ToC (table of contents) APPEARANCE
\usepackage[nottoc,notlof,notlot]{tocbibind} % Put the bibliography in the ToC
\usepackage[titles,subfigure]{tocloft} % Alter the style of the Table of Contents
\renewcommand{\cftsecfont}{\rmfamily\mdseries\upshape}
\renewcommand{\cftsecpagefont}{\rmfamily\mdseries\upshape} % No bold!

%%% END Article customizations

%%% The "real" document content comes below...

\title{HW3}
\author{Wei Ye\footnote{I worked on my assignment sololy. Email: wye22@fordham.edu}  	\\
	ECON 5700}
\date{Due on August 15, 2020.}


\begin{document}
	\maketitle
	\section{Question 1}
	Derive the Taylor expansion for $f(x)=3x^2-6x+5$
	
	\textbf{Solution:}
	
$f_x=6x-6$, and $f_{xx}=6$, but $f_{xxx}=0$, thus, the Taylor Expansion is:
$$f(x)=3x_0^2-6x_0+5+(6x-6)(x-x_0)+3(x-x_0)^2+\textcolor{blue}{o(x-x_0)^n}\footnote{I am not sure whether to add Peano Remainder, so I use blue color to mark. }$$
	
	\section{Question 2}
	Derive the Maclaurin expansion for $e^{kx}$, k is real number. 
	
	\textbf{Solution:}
	
	Let $f(x)=e^{kx}$, then $f_x=ke^{kx}$, $f_{xx}=k^2e^{kx}$, till $f^{(n)}(x)=k^ne^{kx}$.
	\begin{align*}
		f(x)&= 1+kx+\frac{k^2}{2!}x^2+\frac{k^3}{3!}x^3+...+\frac{k^n}{n!}x^n+o(x^n)\\
		&= 1+\sum_{m=1}^{n}\frac{k^m}{m!}x^m +o(x^n)
	\end{align*}
	
	\section{Question 3}
	Derive the Macraulin expansion for $(1+x)^\mu$.
	
	\textbf{Solution:}
	
	Let $f(x)=(1+x)^\mu$, then $f_x=\mu(1+x)^{\mu-1}$, $f_{xx}=\mu(\mu-1)(1+x)^{\mu-2}$.
	\begin{align*}
		f(x)&=1+\mu x+\frac{\mu(\mu-1)}{2!}x^2+\frac{\mu(\mu-1)(\mu-2)}{3!}x^3+...+
		\frac{\mu(\mu-1)...(\mu-n)}{n!}x^n+o(x^n)\\
		&= \sum_{m=0}^{n}\frac{\frac{\mu!}{(\mu-m-1)!}}{m!}x^m+o(x^n)
	\end{align*}	

\section{Question 4}

Derive the Maclaurin expansion for $\sqrt{1+x}$. 

\textbf{Solution:}

Let $f(x)=\sqrt{1+x}$, then $f'(0)=\frac{1}{2}$, $f''(0)=-\frac{1}{4}$, and $f'''(0)=\frac{1}{8}$.
\begin{align*}
	f(x)&=1+\frac{1}{2}x+(-\frac{1}{4})\frac{x^2}{2!}+\frac{3}{8}\frac{x^3}{3!}+...\\
	&= \sum_{m=0}^n \binom{\frac{1}{2}}{m}x^m+o(x^n)	
\end{align*}
		
\section{Question 5}
Find the convexity and concavity for function $f(x)=x^3+ax+b$

\textbf{Solution:}

$f_x=3x^2+a$, $f_{xx}=6x$, thus if $x>0$, it' s convex function, however, it's concave when $x<0$.

\section{Question 6}
Find the intervals of  convexity and concavity of the function $f(x)=\frac{1}{1+x^2}$. 

\textbf{Solution:}

$f_x=\frac{-2x}{(1+x^2)^2}$, and $f_{xx}=\frac{6x^2-2}{(1+x^2)^3}$. Let $f''=0\rightarrow x_1=-\frac{\sqrt{3}}{3}, x_2= \frac{\sqrt{3}}{3}$. Thus, 
when $x<-\frac{\sqrt{3}}{3}$ or $x>\frac{\sqrt{3}}{3}$, it's convex function, otherwise, it's concave. 

\section{Question 7}
Find the intervals of convexity and concavity of the function $f(x)=e^{\frac{1}{x}}$.
	
	\textbf{Solution:}
	
	$f_x=e^{\frac{1}{x}}(-\frac{1}{2})$, and $f_{xx}=e^{\frac{1}{x}}(\frac{2x+1}{x^4})$.
	Thus, if  $x>-\frac{1}{2}$, it's convex function. However, when $x<-\frac{1}{2}$, it's concave function. 
	
\section{Question 8}
Find the sum of the series $S=1-\frac{1}{\sqrt{2}}+\frac{1}{2}-\frac{1}{2\sqrt{2}}+\frac{1}{4}-\frac{1}{4\sqrt{2}}+\frac{1}{8}$.

\textbf{Solution:}


$$S=(1+\frac{1}{2}+\frac{1}{4})(1-\frac{1}{\sqrt{2}})+\frac{1}{8}=\frac{15-7\sqrt{2}}{8}$$

\section{Question 9}
Redo the examples in class on your own: $(1) \sum_{n=1}^{\infty}\frac{3^n}{n^2}$  $(2) \sum_{n=1}^{\infty}\frac{n^3}{(\ln2)^n}$ converges or diverges. 

\textbf{Solution:}
\begin{enumerate}
	\item 
	
	Begin with ratio test.
	
	\begin{align*}
		\lim \frac{\frac{3^{n+1}}{(n+1)^2}}{\frac{3^n}{n^2}}&= 3\lim \frac{n^2}{n^2+1}\\
		&= 3\lim (1-\frac{1}{n+1})^2\\
		&= 3\cdot 1\\
		&=3>1
	\end{align*}
Thus, it's divergent.

\item Begin with Ratio Test again. 

\begin{align*}
	\lim \frac{\frac{(n+1)^3}{(\ln2)^{n+1}}}{\frac{3^n}{(\ln2)^n}}&=\lim \left(\frac{n+1}{n}\right)^3\frac{1}{\ln 2}\\
	&= \lim \left(1+\frac{1}{n}\right)^3\frac{1}{\ln 2}\\
	&= \frac{1}{\ln 2}>1
\end{align*}
Thus, it's divergent	
\end{enumerate}
	
	
	
	
	
\end{document}