%EX TS-program = pdflatex
% !TEX encoding = UTF-8 Unicode

% This is a simple template for a LaTeX document using the "article" class.
% See "book", "report", "letter" for other types of document.

\documentclass[11pt]{article} % use larger type; default would be 10pt

\usepackage[utf8]{inputenc} % set input encoding (not needed with XeLaTeX)

%%% Examples of Article customizations
% These packages are optional, depending whether you want the features they provide.
% See the LaTeX Companion or other references for full information.
\usepackage{amsmath}
\makeatletter
\renewcommand*\env@matrix[1][*\c@MaxMatrixCols c]{%
	\hskip -\arraycolsep
	\let\@ifnextchar\new@ifnextchar
	\array{#1}}
\makeatother

\newcommand{\norm}[1]{\left\lVert#1\right\rVert}
%%% PAGE DIMENSIONS
\usepackage{geometry} % to change the page dimensions
\usepackage{marvosym}
\geometry{a4paper} % or letterpaper (US) or a5paper or....
% \geometry{margin=2in} % for example, change the margins to 2 inches all round
% \geometry{landscape} % set up the page for landscape
%   read geometry.pdf for detailed page layout information

\usepackage{graphicx} % support the \includegraphics command and options
% \usepackage[parfill]{parskip} % Activate to begin paragraphs with an empty line rather than an indent
\usepackage{amssymb}
%%% PACKAGES
\usepackage{booktabs} % for much better looking tables
\usepackage{array} % for better arrays (eg matrices) in maths
\usepackage{paralist} % very flexible & customisable lists (eg. enumerate/itemize, etc.)
\usepackage{verbatim} % adds environment for commenting out blocks of text & for better verbatim
\usepackage{subfig} % make it possible to include more than one captioned figure/table in a single float
% These packages are all incorporated in the memoir class to one degree or another...
\usepackage{pgfplots}
%%% HEADERS & FOOTERS
\usepackage{fancyhdr} % This should be set AFTER setting up the page geometry
\pagestyle{fancy} % options: empty , plain , fancy
\renewcommand{\headrulewidth}{0pt} % customise the layout...
\lhead{}\chead{}\rhead{}
\lfoot{}\cfoot{\thepage}\rfoot{}

%%% SECTION TITLE APPEARANCE
\usepackage{sectsty}
\allsectionsfont{\sffamily\mdseries\upshape} % (See the fntguide.pdf for font help)
% (This matches ConTeXt defaults)
\usepackage[thinc]{esdiff}
\usepackage{bbold}
\usepackage{MnSymbol,wasysym}
%%% ToC (table of contents) APPEARANCE
\usepackage[nottoc,notlof,notlot]{tocbibind} % Put the bibliography in the ToC
\usepackage[titles,subfigure]{tocloft} % Alter the style of the Table of Contents
\renewcommand{\cftsecfont}{\rmfamily\mdseries\upshape}
\renewcommand{\cftsecpagefont}{\rmfamily\mdseries\upshape} % No bold!

%%% END Article customizations

%%% The "real" document content comes below...

\title{Final Exam}
\author{Wei Ye\footnote{I worked on my final exam on my own. There may be many typos, and all the errors are my own.  Email: wye22@fordham.edu}  	\\
	ECON 5700}
\date{Due on August 28, 2021.}

\begin{document}
	\maketitle
	\section{Question 1}
	\textbf{Solution:}

\begin{enumerate}
	\item Rearrange the functions of the question to become:
	$$f_2(x_1,x_2)=2x_1^2+5x_1x_2+2x_2^2$$
	$$\triangledown f_1=(f_{1,x_1},f_{2,x_2})=(-2x_1-\frac{5}{2}x_2, -\frac{5}{2}x_1-2x_2)$$
	$$\triangledown f_2=(f_{2,x_1},f_{2,x_2})=(4x_1+5x_2, 5x_1+4x_2)$$
	\item 
	\begin{align*}
	\mathcal{J}&=\begin{bmatrix}
				-x_1-\frac{5}{2}x_2&-\frac{5}{2}x_1-2x_2\\
				4x_1+5x_2&5x_1+4x_2
			\end{bmatrix}\\
		&=-(5x_1+4x_2)(2x_1+\frac{5}{2}x_2)+(4x_1+5x_2)(\frac{5}{2}x_1+2x_2)\\
	&=	-(10x_1^2+\frac{25}{2}x_1x_2+8x_1x_2+10x_2^2)+(10x_1^2+\frac{25}{2}x_1x_2+8x_1x_2+10x_2^2)\\
		&=0
	\end{align*}
Thus, these two functions are dependent. 

\end{enumerate}

\section{Question 2}
\textbf{Solution:}

$$dz=\frac{\partial z}{\partial x}dx+\frac{\partial z}{\partial y}dy$$
$$\frac{dz}{dt}=\frac{\partial z}{\partial x}\frac{dx}{dt}+\frac{\partial z}{\partial y}\frac{dy}{dt}$$
Thus, we can derive:
\begin{align*}
\frac{dz}{dt}&=(3x^2+8xy-y^2)(-3)+(4x^2-2xy+1)\cdot \frac{1}{2}\\
&=-9x^2-24xy+3y^2+2x^2-xy+\frac{1}{2}\\
&=-7x^2-25xy+3y^2+\frac{1}{2}\\
&=-7(-3t)^2-25(-3t)(1+\frac{1}{2}t)+3(1+\frac{1}{2}t)^2+\frac{1}{2}\\
&=-\frac{9}{4}t^2+78t+\frac{7}{2}
\end{align*}


\section{Quesiton 3}
\textbf{Solution:}

We set up a lagrangian equation from this equation:

$$\mathcal{L}=-c_1^2+c_1c_2-2c_2^2+\lambda(16-c_1-c_2)$$
\begin{equation}
	\frac{\partial \mathcal{L}}{\partial c_1}=-2c_1+c_2-\lambda=0
\end{equation}
\begin{equation}
	\frac{\partial \mathcal{L}}{\partial c_2}=c_1-4c_2-\lambda=0
\end{equation}
\begin{equation}
	\frac{\partial \mathcal{L}}{\partial \lambda}=16-c_1-c_2=0
\end{equation}

Combine the above 3 equations, we can deduce the optimal point of $c_1$ and $c_2$. $c_1^*=10$, $c_2^*=6$. 

$$U_{c_1,c_1}= -2<0$$
$$U_{c_2,c_2}=-4$$
$$U_{c_1,c_2}=1$$
$$D=-2\cdot( -4)-1=7> 0$$

Thus, the points we got are local maximum point. Also, it's easy to justify, this point pair is also global maxmimum point, because put these two point pair to the utility function, and combine with the budget contraint. We can only get the max utility at this point. 

\section{Question 4}
\textbf{Solution:}

Form  Taylor's expansion at $x_0=0$:

\begin{align*}
f(x)&=f(x_0)+f'(x_0)(x-x_0)+\frac{f''(x_0)}{2!}(x-x_0)^2+\frac{f'''(x_0)}{3!}(x-x_0)^3+o(x-x_0)^n\\
&=3+6x+9x^2+12x^3+o(x)^n
\end{align*}

\section{Question 5}
\textbf{Solution:}

\begin{enumerate}
	\item 
	$$\int_{0}^{10} (e^{3x}+4x)dx=\frac{1}{3}e^{3x}\bigg|_0^{10}+2x^2\bigg|_0^{10}=\frac{1}{3}e^{30}+\frac{149}{3}$$
	
	\item 
	$$\int_{0}^{1}6x^2e^{x^3}dx=2\int_{0}^{1}e^{x^3}dx^3=2e^{x^3}\bigg|_0^1=2e-2$$
	
	\item $$\int_{1}^{3x^3}e^tdt=e^{3x^3}-e+c=e^{3x^3}+c\footnote{I do think this method is the same but much easier than the complicated and tedious standard solution. }$$
\end{enumerate}


\section{Question 6}
\textbf{Solution:}

\begin{align*}
	\lim\limits_{x\to 0} &= \lim\limits_{x\to 0} \frac{e^x \sin x +(e^x-1)\cos x}{3x^2+6x}\\
	&= \lim\limits_{x\to 0}\frac{e^x \cos x+e^x \sin x +e^x \cos x -e^x \sin x+\sin x}{6x+6}\\
	&=\lim\limits_{x\to 0} \frac{2e^x \cos x +\sin x}{6x+6}\\
	&=\frac{1}{3}
\end{align*}


\section{Question 7}
\textbf{Solution:}

From the implicit theorem, we can get: 
$$F_x(x,y)dx+F_y(x,y)dy=0$$
From this, we can deduce the function:

$$\frac{dy}{dx}=-\frac{F_x(x,y)}{F_y(x,y)}$$

\begin{align*}
	\frac{dy}{dx}&=-\frac{6x^2-2xy}{-x^2+\frac{1}{y}}\\
	&=\frac{2xy-6x^2}{-x^2+\frac{1}{y}}
\end{align*}

\section{Question 8}
\textbf{Solution:}

\begin{align*}
	\lim\limits_{x\to \infty}\frac{(\frac{1}{2})^{n+2}(n+1)^2}{(\frac{1}{2})^{n+1}n^2}&=
	\lim\limits_{x\to \infty} \frac{1}{2}(1+\frac{1}{n})^2\\
	&=\frac{1}{2}<1
\end{align*}
Thus, it's a convergent sereis. 

\section{Question 9}

\textbf{Solution:}

$$f'(x)=9x^2-4x+1$$
$$f''(x)=18x-4$$
If $f''(x)=0 \longrightarrow x=\frac{2}{9}$, then it would be an inflection point. If $f''(x)<0\longrightarrow \frac{2}{9}$, then it's concavity. If $f''(x)>0\longrightarrow x>\frac{2}{9}$, then it's convexity.

\section{Question 10}

\textbf{Solution:}
\begin{enumerate}
\item 
\begin{itemize}
	\item 
	
	Because $|x-y|\geq 0$, and $1>0$, thus, $\min\{ 1, |x-y|\}\geq 0$ always for any $x,y in \mathcal{R}$. 
	
	\item 
	
	Becaues $d_1(y,x)=\min\{1,|y-x|\}=\min\{1,|x-y|\}=d_1(x,y)$ for any $x,y \in \mathcal{R}$(or say in real field.),
	\item $d_1(x,z)= \min \{1,|x-z|\}\leq \min\{1,|x-y|\}+\min \{1,|y-z|\}$. For this case, it's easy to prove,but I don't know is it required to say specificly. But the main idea is that using y as bridge variable to prove the triangle inequality holds for any $x,y,z \in \mathcal{R}$. 
\end{itemize}

\item 

\begin{enumerate}
	\item \textbf{T}
	\item \textbf{T}
	\item \textbf{T}
	\item \textbf{T}
	\item \textbf{F}
\end{enumerate}

\end{enumerate}

\section{Question 11}
\textbf{Solution:}

\begin{enumerate}
	\item 
	\begin{align*}
		BC^T&=\begin{pmatrix}
			1&3\\3&2
		\end{pmatrix}\begin{pmatrix}
		-1&0&1\\2&1&0
	\end{pmatrix}\\
&=\begin{pmatrix}
	5&3&1\\1&2&3
\end{pmatrix}
	\end{align*}

\item 
\begin{align*}
	\det A&= 1\cdot (2-2)+(-1)^{3+1}(-1-2)\\
	&= -3
\end{align*}

\item 
\begin{align*}
	\det A&= \begin{pmatrix}
		1&0&1\\0&1&3\\0&1&0
	\end{pmatrix}\\
&=\begin{pmatrix}
	1&0&1\\0&1&3\\0&0&-3
\end{pmatrix}\\
&=-3\begin{pmatrix}
	1&0&1\\0&1&3\\0&0&1
\end{pmatrix}\\
&=-3
\end{align*}




\item 
\begin{align*}
	\begin{bmatrix}[ccc|ccc]
		1&0&1&1&0&0\\
		-1&1&2&0&1&0\\
		2&1&2&0&0&1
	\end{bmatrix}&=\begin{bmatrix}[ccc|ccc]
	1&0&1&1&0&0\\
	0&1&3&1&1&0\\
	0&1&0&-2&0&1
\end{bmatrix}\\
&=\begin{bmatrix}[ccc|ccc]
	1&0&1&1&0&0\\
	0&1&3&1&1&0\\
	0&0&-3&-3&-1&1
\end{bmatrix}\\
&=\begin{bmatrix}[ccc|ccc]
	1&0&0&0&-\frac{1}{3}&\frac{1}{3}\\
	0&1&0&-2&0&1\\
	0&0&1&1&\frac{1}{3}&-\frac{1}{3}
\end{bmatrix}
\end{align*}
Thus, $A^{-1}=\begin{bmatrix}
0&-\frac{1}{3}&\frac{1}{3}\\
-2&0&1\\
1&\frac{1}{3}&-\frac{1}{3}
\end{bmatrix}$

$\operatorname{Trace}{(A)}=1+1+2=4 $, From question 3: row operations,  we can easlity get $\operatorname{rank}{(A)}=3$, it's full rank.

\item 
\begin{itemize}
	\item The rank of the matrix is n.
	\item The determinant of this matrix is non-zero, which means the matrix needs to be invertible. 
\end{itemize}
\item 
$$D=\begin{bmatrix}
	2&-3&-1\\1&0&2\\1&1&2
\end{bmatrix}=-5$$
$$D_{x_1}=\begin{bmatrix}
	2&-3&-1\\0&0&2\\1&1&2
\end{bmatrix}=-10$$
$$D_{x_2}=\begin{bmatrix}
	2&2&-1\\1&0&2\\1&1&2
\end{bmatrix}=-5$$
$$D_{x_3}=\begin{bmatrix}
	2&-3&2\\1&0&0\\1&1&1
\end{bmatrix}=5$$
Thus, $x_1=\frac{D_{x_1}}{D}=2$, $x_2=\frac{D_{x_2}}{D}=1$, and $x_3=\frac{D_{x_3}}{D}=-1$. 
\end{enumerate}


\section{Question 12}
\textbf{Solution:}

\begin{enumerate}
	\item We do row echelon of matrix B, thus, we get:
	
	$$B=\begin{bmatrix}
		1&0&1\\
		0&1&-1\\
		0&0&0
	\end{bmatrix}$$
	From the definition of null of matrix, we can get $x_3$ could be any number, $x_2=x_3$, and $x_1=-x_3$.
	One possible basis is $\begin{bmatrix}
		-1\\1\\1
	\end{bmatrix}$

\item $3>0$, and $3\cdot 3-1=8>0$, thus, A is positive definite. 
\item Since the transformation is $T(\vec{x})=A\vec{x}$, thus, the standard matrix of its inverse $T^{-1}$ can be written as $A^{-1}$. 
$$A^{-1}=\frac{1}{8}=\begin{bmatrix}
	3&-1\\-1&3
\end{bmatrix}=\begin{bmatrix}
\frac{3}{8}&-\frac{1}{8}\\-\frac{1}{8}&\frac{3}{8}
\end{bmatrix}$$

\item By reduced row echelon, we can derive B as 
$$\begin{bmatrix}
	1&0&0\\0&1&0\\0&0&1
\end{bmatrix}$$
By the denination of kernal, we need to find $Ax=0$, $Ax_1=0\rightarrow x_1=0$, $Ax_2=0\rightarrow x_2=0$, and $A_2x_3=0\rightarrow x_3=0$. 
Thus, the $\ker (B)=\begin{bmatrix}
	0\\0\\0
\end{bmatrix}$
\item 
\begin{align*}
	|A-\lambda I|&=\begin{bmatrix}
		3-\lambda&1\\1&3-\lambda
	\end{bmatrix}\\
&=(3-\lambda)^2-1\\
&=0
\end{align*}
Thus, $\lambda_1=2, \lambda_2=4$.
\begin{itemize}
	\item When $\lambda=2:$
	$x_1=-x_2$, and the eigenvector is $\begin{bmatrix}
		1\\-1
	\end{bmatrix}$
\item When $\lambda=4:$
$x_1=x_2$, thus, the eigenvector is $\begin{bmatrix}
	1\\1
\end{bmatrix}$
\end{itemize}


\item 
$\operatorname{diag}{(A)}=PDP^{-1}$. 
$$P=\begin{bmatrix}
	1&1\\-1&1
\end{bmatrix}$$
$$D=\begin{bmatrix}
	2&0\\0&4
\end{bmatrix}$$
$$P^{-1}=\begin{bmatrix}
	\frac{1}{2}&-\frac{1}{2}\\\frac{1}{2}&\frac{1}{2}
\end{bmatrix}$$

$$\operatorname{diag}{(A)}=\begin{bmatrix}
	1&1\\-1&1
\end{bmatrix}\begin{bmatrix}
2&0\\0&4
\end{bmatrix}\begin{bmatrix}
\frac{1}{2}&-\frac{1}{2}\\\frac{1}{2}&\frac{1}{2}
\end{bmatrix}=\begin{bmatrix}
3&1\\1&3
\end{bmatrix}$$

\item 
\begin{align*}
A^4&=PD^4P^{-1}\\
&=\begin{bmatrix}
	1&1\\-1&1
\end{bmatrix}\begin{bmatrix}
16&0\\0&256
\end{bmatrix}\begin{bmatrix}
\frac{1}{2}&-\frac{1}{2}\\\frac{1}{2}&\frac{1}{2}
\end{bmatrix}
\end{align*}

\item We let $\vec{x_1}=\begin{bmatrix}
	3\\1
\end{bmatrix}$, and $\vec{x_2}=\begin{bmatrix}
1\\3
\end{bmatrix}$. 

By Grant-Schmidt Process:

$$\vec{v_1}=\vec{x_1}=\begin{bmatrix}
	3\\1
\end{bmatrix}$$
$$\vec{v_2}=\vec{x_2}-(\frac{\vec{v_1}\cdot \vec{x_2}}{\vec{v_1}\cdot \vec{v_1}})\cdot \vec{v_1}=\begin{bmatrix}
	-\frac{4}{5}\\\frac{12}{5}
\end{bmatrix}$$
Normalize the basis:

$$\vec{q_1}=\frac{\vec{v_1}}{\norm{v_1}}=\begin{bmatrix}
	\frac{3}{\sqrt{10}}\\
	\frac{1}{\sqrt{10}}
\end{bmatrix}$$
$$\vec{q_2}=\frac{\vec{v_2}}{\norm{v_2}}=\begin{bmatrix}
	-\frac{\sqrt{10}}{10}\\\frac{3\sqrt{10}}{10}
\end{bmatrix}$$

\item 
From (h)/(8), we can get the Q:

$$Q=\begin{bmatrix}
	\frac{3\sqrt{10}}{10}&-\frac{\sqrt{10}}{10}\\
	\frac{\sqrt{10}}{10}&\frac{3\sqrt{10}}{10}
\end{bmatrix}$$

Since $A=QR\longrightarrow Q^TA=Q^TQR\rightarrow R=Q^TA$. 
$$R=Q^TA=\begin{pmatrix}
\frac{3\sqrt{10}}{10}&\frac{\sqrt{10}}{10}\\
-\frac{\sqrt{10}}{10}&\frac{3\sqrt{10}}{10}
\end{pmatrix}\begin{pmatrix}
3&1\\1&3
\end{pmatrix}=\begin{pmatrix}
\frac{2\sqrt{10}}{5}&\frac{3\sqrt{10}}{5}\\0&\frac{4\sqrt{10}}{5}
\end{pmatrix}$$

\item 

\begin{align*}
A^+&=(A^TA)^{-1}A^T\\
&=(\begin{pmatrix}
	3&1\\1&3
\end{pmatrix}\begin{pmatrix}
3&1\\1&3
\end{pmatrix})^{-1}\begin{pmatrix}
3&1\\1&3
\end{pmatrix}\\
&=\begin{pmatrix}
	10&6\\6&10
\end{pmatrix}^{-1}\begin{pmatrix}
3&1\\1&3
\end{pmatrix}\\
&=\frac{1}{64}\begin{pmatrix}
	10&-6\\-6&10
\end{pmatrix}\begin{pmatrix}
3&1\\1&3
\end{pmatrix}\\
&=\begin{pmatrix}
	\frac{3}{8}&-\frac{1}{8}\\
	-\frac{1}{8}&\frac{3}{8}
\end{pmatrix}
\end{align*}


\end{enumerate}

\section{Question 13}
\begin{enumerate}
	\item \textbf{F}
	\item \textbf{T}
	\item \textbf{F}
	\item \textbf{T}
	\item \textbf{T}
	\item \textbf{T}
	\item \textbf{F}
\end{enumerate}

\section{Question 14}
\textbf{Solution:}
\begin{enumerate}
	\item 
$$\begin{pmatrix}
	3&2&0\\0&1&3\\1&4&1
\end{pmatrix}\xrightarrow{R_3-\frac{1}{3}R_1}\begin{pmatrix}
3&2&0\\0&1&3\\0&\frac{10}{3}&1
\end{pmatrix}\xrightarrow{R_3-\frac{10}{3}R_2}\begin{pmatrix}
3&2&0\\0&1&3\\0&0&-9
\end{pmatrix}=U$$
$$L=\begin{pmatrix}
	1&0&0\\0&1&0\\\frac{1}{3}&\frac{10}{3}&1
\end{pmatrix}$$


\item 

First, we need to find the eigenvalues of matrix A:

\begin{align*}
	|A-\lambda I|&=\begin{vmatrix}
		3-\lambda&2&0\\
		0&1-\lambda&3\\
		1&4&1-\lambda
	\end{vmatrix}\\
&=(1-\lambda)((1-\lambda)(3-\lambda))-3(4(3-\lambda)-2)\\
&=-\lambda^3+5\lambda^2+5\lambda-27\\
&=0
\end{align*}
Thus, from above we can get \footnote{I use online calculator to compute the eigenvalues, because it's impossible to compute these eigenvaluse manually. } $\lambda_1=4.89449$, $\lambda_2=-2.29654$, and $\lambda_3=2.40205$.

Thus:

$$D=\begin{bmatrix}
	4.89449&0&0\\0&-2.29654&0\\0&0&2.40205
\end{bmatrix}$$

$$LDL^T=\begin{bmatrix}
	1&0&0\\0&1&0\\\frac{1}{3}&\frac{10}{3}&1
\end{bmatrix}\begin{bmatrix}
	4.89449&0&0\\0&-2.29654&0\\0&0&2.40205
\end{bmatrix}\begin{bmatrix}
1&0&\frac{1}{3}\\0&1&\frac{10}{3}\\0&0&1
\end{bmatrix}$$

\item 
From 2, we can get there is no cholesky decomposition with respect to A.
Because 

$$D=dd$$

However, we can't get square root of negative eigenvalue, so the cholesky decomposition doesn't exist in this case. 

\end{enumerate}

\section{Bonus Questions}
\textbf{Solution:}

\begin{enumerate}
	\item 
	$$\operatorname{adj}{B}=\begin{bmatrix}
		2&3\\3&1
	\end{bmatrix}$$.
\item 
Use the results of question 12 directly, the spectrum decompose of A:
\begin{align*}
	\operatorname{Decompose}{A}&=QDQ^T\\
	&=\begin{bmatrix}
		\frac{3\sqrt{10}}{10}&-\frac{\sqrt{10}}{10}\\
		\frac{\sqrt{10}}{10}&\frac{3\sqrt{10}}{10}
	\end{bmatrix}\begin{bmatrix}
	2&0\\0&4
\end{bmatrix}\begin{bmatrix}
\frac{3\sqrt{10}}{10}&\frac{\sqrt{10}}{10}\\
-\frac{\sqrt{10}}{10}&\frac{3\sqrt{10}}{10}
\end{bmatrix}
\end{align*}

\item 
\begin{enumerate}
	\item $$A^TA=\begin{pmatrix}
		1&0\\0&2\\1&0
	\end{pmatrix}\begin{pmatrix}
	1&0&1\\0&2&0
\end{pmatrix}=\begin{pmatrix}
1&0&1\\0&4&0\\1&0&1
\end{pmatrix}$$

\begin{align*}
	|A^TA-\lambda I|&=\begin{vmatrix}
		1-\lambda&0&1\\
		0&4-\lambda&0\\
		1&0&1-\lambda
	\end{vmatrix}\\
	&=(4-\lambda)[(1-\lambda)^2-1]\\
	&=-\lambda^3+6\lambda^2-8\lambda\\
	&=0
\end{align*}

Thus,  $\lambda_1=4$, $\lambda_2=0$, and $\lambda_3=2$. $\sigma_1=2, \sigma_2=0, \sigma_3=\sqrt{2}$. 
$$\Sigma=\begin{bmatrix}
	2&0&0\\0&\sqrt{2}&0
\end{bmatrix}$$
\item
\begin{itemize}
	\item When $\lambda=4$, the eigenvector is $\begin{pmatrix}
		0\\1\\0
	\end{pmatrix}$
\item When $\lambda=2$, the eigenvector is $\begin{pmatrix}
	1\\0\\1
\end{pmatrix}$
\item When $\lambda=0$, the eigenvector is $\begin{pmatrix}
	1\\0\\-1
\end{pmatrix}$
 
 By G-S decomposition:
 
$$V=\begin{pmatrix}
	0&1&1\\1&0&0\\0&1&-1
\end{pmatrix}$$





 $$Q=\begin{bmatrix}
 	0&	\frac{\sqrt{2}}{2}&	\frac{\sqrt{2}}{2}\\
 	1&0&0\\
 	0&	\frac{\sqrt{2}}{2}&-\frac{\sqrt{2}}{2}\\
 \end{bmatrix}$$
\end{itemize}
\item 
$$\vec{U_1}=\frac{1}{2}Aq_1=\begin{bmatrix}
	0\\1
\end{bmatrix}$$
$$\vec{U_2}=\frac{1}{\sqrt{2}}\begin{pmatrix}
	1&0&1\\0&2&0
\end{pmatrix}\begin{pmatrix}
	\frac{\sqrt{2}}{2}\\0\\	\frac{\sqrt{2}}{2}
\end{pmatrix}=\begin{pmatrix}
1\\0
\end{pmatrix}$$
Thus, $U=\begin{pmatrix}
	0&1\\1&0
\end{pmatrix}$
\end{enumerate}



\end{enumerate}
























































































\end{document}


